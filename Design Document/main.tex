\documentclass[12pt,a4paper]{article}

\usepackage{graphicx}
\usepackage{float}
\usepackage{hyperref}
\usepackage{geometry}
\geometry{margin=1in}
\documentclass{article}
\usepackage[utf8]{inputenc}
\usepackage{graphicx}  % for resizebox




\begin{document}


\newpage

%-------------------------------------------------
\begin{titlepage}
\centering

% University Logo
\vspace*{1cm}
\includegraphics[width=0.25\textwidth]{namal.png}

\vspace{1cm}

% University Name
{\Large \textbf{Namal University, Mianwali}}\\[0.3cm]
{\large Department of Computer Science}

\vspace{1.2cm}

% Course Name
{\large \textbf{Course Name:}}\\
{\large CSC-225 – Software Engineering} % <-- change if needed

\vspace{1.5cm}

% Title
{\Huge \textbf{System Design Document}}\\[0.3cm]
{\Large AI-Based Task Manager \& Smart Scheduler}

\vspace{1.5cm}
        {\Large Version 1.0}\\
        {\vspace{0.2cm}}
        {\large Approved}
        \vspace{1.5cm}

% Prepared By
{\large \textbf{Prepared By}}\\[0.4cm]

\begin{tabular}{l}
Mishaal Aqdas \hspace{1cm} (NUM-BSCS-2024-35)\\
Muhammad Imran \hspace{0.6cm} (NUM-BSCS-2024-50)\\
Khadija Tul Kubra \hspace{0.5cm} (NUM-BSCS-2024-30)
\end{tabular}

\vspace{1.5cm}

% Submitted To
{\large \textbf{Submitted To}}\\[0.3cm]
{\large Mam Asiya Batool}

\vfill

% Date
{\large \textbf{Date:} 18 January 2026}

\end{titlepage}
\newpage

\tableofcontents
\newpage
\section{Introduction}

The purpose of this design report is to present the system design for the \textbf{AI-Based Task Manager \& Smart Scheduler} as part of Project Milestone 3. This document translates the approved Software Requirements Specification (SRS) into a structured design that defines system behavior, constraints, and major design decisions prior to implementation. The design focuses on clarity, modularity, and feasibility to ensure smooth development and future scalability.

%-------------------------------------------------

\section{Design Assumptions and Constraints}

\subsection{Design Assumptions}
\begin{itemize}
    \item Users will have access to smartphones with basic internet connectivity.
    \item The system will be used primarily by individual users for personal task management.
    \item AI-based scheduling decisions are based on historical task data and user preferences.
    \item External services such as notification and calendar APIs will be available and reliable.
\end{itemize}
%--------------------------------------
\subsection{Design Constraints}
\begin{itemize}
    \item Limited computational resources on mobile devices.
    \item Dependency on third-party APIs for notifications and calendar integration.
    \item Offline functionality is restricted to cached data only.
    \item Security and privacy constraints for storing user task data.
\end{itemize}

%-------------------------------------------------
\section{Key Design Decisions}

Several key design decisions were made to ensure system efficiency and maintainability:
\begin{itemize}
    \item A layered architecture was selected to separate concerns and simplify maintenance.
    \item AI logic was isolated into a dedicated layer to allow independent enhancement of intelligent features.
    \item Modular services were used to support scalability and future feature expansion.
    \item Cloud-based storage was chosen to enable data backup and synchronization across devices.
\end{itemize}


%-------------------------------------------------
\subsection{Figma Prototyping}

Figma was used to design and prototype the system diagrams, including Data Flow Diagrams (DFDs). 
The interactive prototype helps visualize system processes and validate design decisions before implementation.

\noindent
\textbf{Prototype Link:}  \\
\href{https://www.figma.com/design/fKUv7xkuMAmSGAMIT4smiD/Ai-based-smart-schedular?node-id=0-1&t=h985N7fXUsmoqRSM-1}{Figma System Design Prototype}

%--------------------------------------------
\section{Use Case Design Description}

Use case design identifies system actors and major functionalities.  
Registered users manage tasks, schedules, notifications, and progress.  
System administrators manage users and monitor system operations.

\begin{figure}[H]
\centering
\includegraphics[width=0.85\textwidth]{use_case.png}
\caption{Use Case Diagram}
\label{fig:usecase}
\end{figure}
\textbf{Link:}  
\href{https://www.figma.com/design/Jzr6qsU2eSbA5oKBjqp0l4/Untitled?node-id=1-2&t=uzO61Erf4177GbQJ-1}{Use Case Diagram Link}
\\
\textbf{Diagram Instruction:}  
Design a UML Use Case Diagram showing:
\begin{itemize}
    \item Actor: User, Admin
    \item Use cases: Create Task, Schedule Task, Receive Reminder, AI Reschedule
\end{itemize}

%-------------------------------------------------

\section{Data Flow Design Description}

Data Flow Diagrams (DFDs) represent how data moves through the system.

\subsection{Level 0 DFD}

\begin{figure}[H]
\centering
\includegraphics[width=0.85\textwidth]{context.png}
\caption{DFD Level 0}
\label{fig:dfd0}
\end{figure}
\textbf{Link:}  
\href{https://www.figma.com/design/Jzr6qsU2eSbA5oKBjqp0l4/Untitled?node-id=1-2&t=uzO61Erf4177GbQJ-1}{DFD 0 Diagram Link}
\subsection{Level 1 and Level 2 DFD}

\begin{figure}[H]
\centering
\includegraphics[width=0.85\textwidth]{dfd1.png}
\caption{DFD Level 1}
\label{fig:dfd1}
\end{figure}

\begin{figure}[H]
\centering
\includegraphics[width=0.85\textwidth]{dfd2.png}
\caption{DFD Level 2}
\label{fig:dfd2}
\end{figure}

\textbf{Diagram Instruction:}  \\
DFD is created using:
\begin{itemize}
    \item figma
\end{itemize}
\vspace{.1cm}
{\large \textbf{DFD 1 and 2  Diagram Link:}}\\
\href{https://www.figma.com/design/i1IUNTzwPR4jTwERnGFFdA/dfd?node-id=28-2&t=6xSmsDsGcxrTb0yW-1}{\texttt{Figma Design - DFD1 Diagrams}}
%-------------------------------------------------
\newpage
\section{Sequence Diagram}

Sequence diagrams model interactions between users, system components, databases, and external services.

\begin{figure}[H]
\centering
\includegraphics[width=0.85\textwidth]{sequence1.png}
\includegraphics[width=0.85\textwidth]{sequence2.png}
\end{figure}
\begin{figure}[H]
\includegraphics[width=0.85\textwidth]{sequence3.png}
\includegraphics[width=0.85\textwidth]{sequence4.png}
\caption{Sequence Diagram }
\label{fig:sequence}
\end{figure}

\textbf{Diagram Instruction:}  
Include lifelines for:
\begin{itemize}
    \item User
    \item Mobile App
    \item AI Engine
    \item Database
    \item Notification Service
\end{itemize}

%-------------------------------------------------
\newpage
\section{Workflow Design Description}

Activity diagrams illustrate workflows such as task creation, AI rescheduling, notifications, and backup.

\begin{figure}[H]
\centering
\includegraphics[width=0.85\textwidth]{activity1.jpeg}
\end{figure}
\begin{figure}[H]
\includegraphics[width=0.85\textwidth]{activity2.jpeg}
\end{figure}
\begin{figure}[H]
\includegraphics[width=0.85\textwidth]{activity3.jpeg}
\caption{Activity Diagram }
\label{fig:activity}
\end{figure}

%-------------------------------------------------

\section{Structural and Component Design Description}

\subsection{Class Diagram}

\begin{figure}[H]
\centering
\includegraphics[width=0.85\textwidth]{uml.png}
\caption{Class Diagram}
\label{fig:class}
\end{figure}

\subsection{Component Diagram}

\begin{figure}[H]
\centering
\includegraphics[width=0.85\textwidth]{component.png}
\caption{Component Diagram}
\label{fig:component}
\end{figure}
{\large \textbf{Link:}}\\
\href{https://www.figma.com/design/i1IUNTzwPR4jTwERnGFFdA/dfd?node-id=28-2&t=6xSmsDsGcxrTb0yW-1}{\texttt{Component Diagram Link}}
%-------------------------------------------------
\section{GitHub Link}

The system design artifacts and prototypes are maintained using modern collaboration tools:
\begin{itemize}
    \item \textbf{GitHub} is used for version control and managing project documentation.
\end{itemize}

\noindent
\textbf{Link:}\\
\href{https://github.com/bscs24f35-cell/AI-Based-Task-Manager-Smart-Scheduler.git}{AI-Based Task Manager -- GitHub repository Link}
%---------------------------------------------------------

%-------------------------------------------------

\section{Conclusion}

This system design provides a structured blueprint for implementing the AI-Based Task Manager \& Smart Scheduler.  
By aligning design artifacts with requirements and validating them through prototypes, the system is well-prepared for the implementation phase.

\end{document}
